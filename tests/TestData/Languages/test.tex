% LaTeX Test File for UAST-Grep
% Tests: document structure, commands, environments, math

\documentclass[12pt,a4paper]{article}

% Package imports
\usepackage[utf8]{inputenc}
\usepackage[T1]{fontenc}
\usepackage{amsmath,amssymb,amsthm}
\usepackage{graphicx}
\usepackage{hyperref}
\usepackage{listings}
\usepackage{xcolor}
\usepackage{geometry}
\usepackage{fancyhdr}
\usepackage{tikz}
\usepackage{algorithm}
\usepackage{algpseudocode}

% Page geometry
\geometry{margin=1in}

% Custom colors
\definecolor{codegreen}{rgb}{0,0.6,0}
\definecolor{codegray}{rgb}{0.5,0.5,0.5}
\definecolor{codepurple}{rgb}{0.58,0,0.82}
\definecolor{backcolour}{rgb}{0.95,0.95,0.92}

% Listings configuration
\lstdefinestyle{mystyle}{
    backgroundcolor=\color{backcolour},
    commentstyle=\color{codegreen},
    keywordstyle=\color{magenta},
    numberstyle=\tiny\color{codegray},
    stringstyle=\color{codepurple},
    basicstyle=\ttfamily\footnotesize,
    breakatwhitespace=false,
    breaklines=true,
    captionpos=b,
    keepspaces=true,
    numbers=left,
    numbersep=5pt,
    showspaces=false,
    showstringspaces=false,
    showtabs=false,
    tabsize=2
}
\lstset{style=mystyle}

% Custom commands
\newcommand{\uastgrep}{\textsc{UAST-Grep}}
\newcommand{\version}{1.0.0}
\newcommand{\email}[1]{\href{mailto:#1}{\texttt{#1}}}
\newcommand{\code}[1]{\texttt{#1}}

% Custom commands with parameters
\newcommand{\highlight}[1]{\colorbox{yellow}{#1}}
\newcommand{\important}[1]{\textbf{\textcolor{red}{#1}}}

% Renewcommand
\renewcommand{\abstractname}{Executive Summary}

% New environment
\newenvironment{note}
    {\begin{quote}\itshape\textbf{Note:}}
    {\end{quote}}

% Theorem environments
\newtheorem{theorem}{Theorem}[section]
\newtheorem{lemma}[theorem]{Lemma}
\newtheorem{proposition}[theorem]{Proposition}
\newtheorem{corollary}{Corollary}[theorem]
\newtheorem{definition}{Definition}[section]

% Counter
\newcounter{example}[section]
\newenvironment{example}[1][]
    {\refstepcounter{example}\par\medskip\textbf{Example~\theexample. #1}\rmfamily}
    {\medskip}

% Header and footer
\pagestyle{fancy}
\fancyhf{}
\lhead{\uastgrep{} Documentation}
\rhead{Version \version}
\cfoot{\thepage}

% Title information
\title{\uastgrep{}: Cross-Language AST Grep Tool}
\author{Test Author\thanks{Supported by UAST-Grep Foundation} \\
        \email{test@example.com}}
\date{\today}

\begin{document}

\maketitle

\begin{abstract}
This document demonstrates \LaTeX{} features for testing the \uastgrep{} parser.
It includes various elements such as sections, equations, tables, figures,
and custom commands.
\end{abstract}

\tableofcontents
\listoffigures
\listoftables

\newpage

\section{Introduction}
\label{sec:intro}

\uastgrep{} is a cross-language AST grep tool supporting 74 languages
with a unified schema. This document serves as a \LaTeX{} test file
for the parser.

\subsection{Features}

The main features include:
\begin{itemize}
    \item Cross-language AST parsing
    \item Unified AST schema
    \item Pattern matching with queries
    \item High-performance native bindings
\end{itemize}

\begin{note}
This is a custom note environment demonstrating environment creation.
\end{note}

\section{Mathematical Expressions}
\label{sec:math}

\subsection{Inline Math}

The quadratic formula is $x = \frac{-b \pm \sqrt{b^2 - 4ac}}{2a}$.

Greek letters: $\alpha, \beta, \gamma, \delta, \epsilon, \theta, \lambda, \mu, \pi, \sigma, \omega$.

\subsection{Display Math}

The Gaussian integral:
\begin{equation}
    \int_{-\infty}^{\infty} e^{-x^2} \, dx = \sqrt{\pi}
    \label{eq:gaussian}
\end{equation}

Equation~\ref{eq:gaussian} is fundamental in probability theory.

\subsection{Aligned Equations}

\begin{align}
    (a + b)^2 &= a^2 + 2ab + b^2 \label{eq:binomial1} \\
    (a - b)^2 &= a^2 - 2ab + b^2 \label{eq:binomial2} \\
    (a + b)(a - b) &= a^2 - b^2 \label{eq:binomial3}
\end{align}

\subsection{Matrix}

\begin{equation}
    A = \begin{pmatrix}
        a_{11} & a_{12} & a_{13} \\
        a_{21} & a_{22} & a_{23} \\
        a_{31} & a_{32} & a_{33}
    \end{pmatrix}
\end{equation}

\subsection{Theorems}

\begin{definition}
A \textbf{processor} is a function $P: \mathcal{I} \to \mathcal{O}$
that transforms input items to output items.
\end{definition}

\begin{theorem}[Transformation Property]
\label{thm:transform}
For any item $x \in \mathcal{I}$, the transformation $T(x)$ satisfies:
\begin{enumerate}
    \item If $x > 0$, then $T(x) = 2x$
    \item If $x < 0$, then $T(x) = -x$
    \item If $x = 0$, then $T(x) = 0$
\end{enumerate}
\end{theorem}

\begin{proof}
The proof follows directly from the definition of $T$.
\end{proof}

\section{Tables and Figures}
\label{sec:tables}

\subsection{Tables}

\begin{table}[htbp]
    \centering
    \caption{Supported Languages by Tier}
    \label{tab:languages}
    \begin{tabular}{|c|l|l|c|}
        \hline
        \textbf{ID} & \textbf{Language} & \textbf{Extension} & \textbf{Tier} \\
        \hline
        1 & TypeScript & .ts & 1 \\
        2 & Python & .py & 1 \\
        3 & Rust & .rs & 1 \\
        4 & Go & .go & 1 \\
        \hline
    \end{tabular}
\end{table}

See Table~\ref{tab:languages} for the language list.

\subsection{Figures}

\begin{figure}[htbp]
    \centering
    \begin{tikzpicture}
        \draw[thick,->] (0,0) -- (4,0) node[anchor=north] {$x$};
        \draw[thick,->] (0,0) -- (0,3) node[anchor=east] {$y$};
        \draw[blue,thick] (0,0) parabola (3,2);
        \node at (2,1) {$f(x)$};
    \end{tikzpicture}
    \caption{Example TikZ diagram}
    \label{fig:tikz}
\end{figure}

\section{Code Listings}
\label{sec:code}

\begin{lstlisting}[language=Python, caption=Python Example]
def transform(item):
    """Transform an item based on its value."""
    if item > 0:
        return item * 2
    elif item < 0:
        return -item
    else:
        return 0

# Main execution
result = transform(42)
print(f"Result: {result}")
\end{lstlisting}

\section{Algorithms}
\label{sec:algo}

\begin{algorithm}
\caption{Process Items}
\label{alg:process}
\begin{algorithmic}[1]
\Require Items array $I$
\Ensure Processed results $R$
\State $R \gets \emptyset$
\For{each $item$ in $I$}
    \State $result \gets \Call{Transform}{item}$
    \State $R \gets R \cup \{result\}$
\EndFor
\State \Return $R$

\Function{Transform}{$x$}
    \If{$x > 0$}
        \State \Return $2x$
    \ElsIf{$x < 0$}
        \State \Return $-x$
    \Else
        \State \Return $0$
    \EndIf
\EndFunction
\end{algorithmic}
\end{algorithm}

\section{Cross-References}
\label{sec:refs}

As shown in Section~\ref{sec:intro}, the \uastgrep{} tool is powerful.
Theorem~\ref{thm:transform} describes the transformation property.
See Figure~\ref{fig:tikz} and Algorithm~\ref{alg:process}.

\section{Bibliography}

\begin{thebibliography}{9}
\bibitem{knuth}
    Donald E. Knuth,
    \textit{The Art of Computer Programming},
    Addison-Wesley, 1997.

\bibitem{lamport}
    Leslie Lamport,
    \textit{\LaTeX: A Document Preparation System},
    Addison-Wesley, 1994.
\end{thebibliography}

\appendix
\section{Additional Information}
\label{app:info}

This appendix contains supplementary material.

\begin{example}[Basic Usage]
Here is an example of using the \uastgrep{} tool:
\begin{verbatim}
UAST-Grep --pattern "function $NAME" --lang typescript ./src
\end{verbatim}
\end{example}

\end{document}
